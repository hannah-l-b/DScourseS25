\documentclass[12pt,english]{article}
\usepackage{mathptmx}

\usepackage{color}
\usepackage[dvipsnames]{xcolor}
\definecolor{darkblue}{RGB}{0.,0.,139.}

\usepackage[top=1in, bottom=1in, left=1in, right=1in]{geometry}

\usepackage{amsmath}
\usepackage{amstext}
\usepackage{amssymb}
\usepackage{setspace}
\usepackage{lipsum}
\usepackage{tabularx}


\usepackage{natbib}
\usepackage{url}
\usepackage{booktabs}
\usepackage[flushleft]{threeparttable}
\usepackage{graphicx}
\usepackage[english]{babel}
\usepackage{pdflscape}
\usepackage[unicode=true,pdfusetitle,
 bookmarks=true,bookmarksnumbered=false,bookmarksopen=false,
 breaklinks=true,pdfborder={0 0 0},backref=false,
 colorlinks,citecolor=black,filecolor=black,
 linkcolor=black,urlcolor=black]
 {hyperref}
\usepackage[all]{hypcap} % Links point to top of image, builds on hyperref
\usepackage{breakurl}    % Allows urls to wrap, including hyperref

\linespread{2}

\begin{document}

\begin{singlespace}
\title{Mindsets in Motion: The Role of Priming and Sequence in Professional Skepticism}
\end{singlespace}

\author{Hannah Bermudez\thanks{Department of Accounting, University of Oklahoma.\
E-mail~address:~\href{mailto:student.name@ou.edu}{hlbermudez@ou.edu}}}

% \date{\today}
\date{April 22, 2025}

\maketitle

\begin{abstract}
\begin{singlespace}
This study examines how auditors revise their beliefs in response to sequentially presented audit evidence and whether priming affects this revision process. Drawing on the belief-adjustment model, I explore the cognitive mechanisms underlying auditor judgment, including anchoring, recency effects, and susceptibility to management cues. Through an experimental design, participants received audit evidence in varying orders and were exposed to either a management-supportive prime or no prime. Results show that evidence order significantly influences belief revision, with auditors more sensitive to later-presented information. Moreover, primed participants demonstrated more conservative belief updating, contrary to expectations. These findings suggest that both evidence presentation and psychological framing meaningfully shape audit judgment, highlighting the need for audit methodologies that mitigate cognitive bias. The study contributes to the literature on behavioral auditing by identifying conditions under which professional skepticism may be compromised and provides practical implications for training and audit documentation standards
\end{singlespace}

\end{abstract}
\vfill{}


\pagebreak{}


\section{Introduction}\label{sec:intro}
\subsection{Outline}

\begin{itemize}
    \item \textbf{Auditor Role in Evidence Evaluation} \\
    Auditors must interpret and integrate diverse information to form professional judgments. Judging conflicting or uncertain evidence is cognitively demanding.

    \item \textbf{Limitations of Normative Models} \\
    Bayes’ Theorem has historically guided belief revision. However, it falls short in explaining how people intuitively update beliefs.

    \item \textbf{Psychological Models of Belief Updating} \\
    \citet{hogarth1992} belief-adjustment model accounts for order effects and anchoring. Auditors revise beliefs sequentially, not holistically.

    \item \textbf{Evidence from Auditing Research} \\
    Studies \citep{ashton1988, messier1994} show auditors often rely on additive or heuristic processing. Belief revision is sensitive to the presentation and framing of evidence.

    \item \textbf{Gap in the Literature} \\
    Limited research on how multiple cognitive factors interact (e.g., priming, skepticism, recency). There is a need to understand how real-world conditions influence auditors’ evidence interpretation.

    \item \textbf{Study Motivation} \\
    This study explores how priming and evidence sequencing affect auditor belief revision. It aims to identify when auditors are most susceptible to judgment bias and how to mitigate it.
\end{itemize}

\section{Background and Hypothesis Development}
\subsection{Belief Revision}
\hspace*{2em} Bayes’ Theorem was once the dominant normative model of belief revision and played a central role in studies on audit evidence evaluation. However, research has since found the model to be incomplete as a descriptive account of belief revision due to its inability to adequately predict intuitive updates \citep{ashton1988}. For example, \citet{pitz1967} demonstrated that individuals’ tendency to revise their subjective probabilities is often influenced by task characteristics, highlighting the theory’s limitations in capturing real-world decision-making.
In response to these limitations, \citet{hogarth1992} proposed a model that assumes that individuals update their beliefs through a sequential anchoring-and-adjustment process. In this process, a current opinion is adjusted based on new evidence, and the revised belief then serves as the new anchor for subsequent adjustments. This process allows (1) belief revision to be influenced by the order in which information is evaluated, (2) the extent of belief revision to depend on the decision-maker’s sensitivity to the evidence, (3) the size of the anchor to affect the revision, and (4) the decision-maker to increase or decrease support for a hypothesis \citep{krishnamoorthy1999}. These findings indicate that auditors and other decision-makers do not update beliefs in a strictly Bayesian manner but are instead influenced by task structure, prior beliefs, and cognitive biases. \citet{messier1994} substantiate this assessment by finding that auditors typically follow an additive model when revising their beliefs, first assessing an item’s relation to an audit assertion before making adjustments. Understanding how auditors revise their beliefs is critical for assessing the reliability of their judgments and the potential biases that may arise in audit decision-making.
\hspace*{2em} However, an important yet underexplored aspect of belief revision is the interaction of multiple cognitive effects. \citet{hogarth1992} explicitly emphasized that order effects depend on procedural and contextual variables and called for further research on how task-specific factors influence belief revision. While extensive research in psychology and accounting has confirmed the existence of presentation order effects and how individuals encode and process evidence, no study to date to my knowledge has examined the combined effects of presentation order, sequential processing, recency effects, priming, and individual auditor traits. Prior research establishes that belief revision is not solely a function of evidence evaluation but also of how auditors process and integrate information over time. Addressing these factors is essential for a deeper understanding of how auditors weigh and interpret information in real-world settings, which could enhance audit methodologies and mitigate bias.

\subsection{Malleability of Beliefs}
\hspace*{2em} A simplified experimental setting, such as the one presented in the \citet{ashton1988} study, fails to fully capture the complexity of the environment auditors must navigate when making informed judgments. Auditors must not only evaluate the sufficiency and appropriateness of audit evidence but also assess compliance with GAAP, quantitative and qualitative materiality considerations, management representations, the reasonableness of estimates and judgments, and the quality of disclosures—all while maintaining professional skepticism and integrating these factors holistically. While an experiment cannot fully encapsulate the entire audit process, it can provide valuable insights into specific aspects of auditor decision-making and the factors that influence their judgments. 

Prior studies have largely examined belief revision in isolation \citep{ashton1988, pei1992, kennedy1995}. However, exploring additional interacting factors could provide a more comprehensive understanding of how auditors form and revise their judgments. Since belief revision is a dynamic process influenced by new evidence, understanding the external factors that shape this process is crucial. Priming, as one such factor, may subtly bias how auditors update their assessments. And given that belief revision depends on the salience and order of presented evidence, it makes sense that factors such as priming may introduce biases, shaping how auditors update their assessments. One particularly influential factor in this process is management’s claims, which can serve as cognitive cues that subtly prime auditors to focus on specific aspects of an audit while overlooking others. As a result, auditors’ judgments may become unintentionally aligned with management’s representations.

The term priming refers to the phenomenon where exposure to a stimulus influences how someone responds to subsequent stimuli. In this context, an individual’s existing beliefs play a significant role in shaping their perceptions, while external cues can disproportionately influence how these judgments are formed. Understanding the mechanisms that shape an individual's initial perceptions is crucial, as these perceptions influence both preconceptions and subsequent judgments.

William James’ concept of ideo-motor action (1890) aligns with modern theories of cognitive priming. He posited that merely thinking about a behavior facilitates engagement in that behavior, as cognitive activation makes related concepts more accessible. This principle underlies the priming effect, where prior exposure to a stimulus influences subsequent judgments and actions unconsciously. That is that the cognitive process of thinking about something activates the corresponding concept in your mind, making it more accessible and easier to use. Therefore, James suggests that there is a direct connection between thought and behavior, and that a priming effect exists through the automaticity of cognitive processes which can unconsciously influence actions. This includes judgments and decision-making.

There is a considerable amount of research that supports this premise. For example, \citet{loersch2016} posit that “...information made accessible by priming drives an inference process in which it is used as evidence for target-related judgments” (p.6). This suggests that primed information serves as a reference point in judgment formation, becoming the most salient piece of information when evaluating subsequent evidence. Similarly, \citet{bargh1996} found that activation of stimuli by priming can unconsciously shift behavior to align with the primed concept. Together, these findings demonstrate that exposure to certain cues can shape both thought processes and actions. 

Returning to the audit context, as has already been established, auditors are expected to thoroughly evaluate all pieces of evidence to form an opinion. However, the discussion above highlights the possibility that certain pieces of evidence could unintentionally anchor an auditor’s judgment or subconsciously lead them to seek out evidence supporting a primed conclusion. Ashton and Ashton did not account for this when examining the impact of the timing and nature of the evidence presented, nor were they able to do so within the constraints of their experimental setting. Therefore, I anticipate that priming will have a significant effect on how auditors revise and update their beliefs, particularly when considered alongside task characteristics and recency effects. 

First, research has shown that affective reactions can occur with minimal stimulation and subsequently influence cognition \citep{zajonc1980, murphy1993}, suggesting that even subtle cues can bias judgement. For instance, research has shown that priming can subtly influence various judgments, from implicit gender stereotypes \citep{rudman2010} to perceptions of AI trustworthiness \citep{pataranutaporn2023} and investor decision-making \citep{kilger2012}. Similarly, in audit settings, priming has been found to shape auditors' self-awareness \citep{bambani2017} and risk assessment judgments \citep{hammersley2010}. Thus, a significant primacy effect may have been present in Ashton and Ashton’s experiment but went undetected in their study.

Second, the recency effect is much more dominant and may be the reason why \citet{ashton1988} findings suggest that auditors are more influenced by the most recently presented information when forming judgments. \citet{murdock1962} examined how the position of an item in a list influences the likelihood of its recall and found that the recency effect can be more pronounced when recall happens shortly after presentation. Similarly, \citet{daniel2018} found that perception in a serial-position-recognition task follows a recency-to-primacy, where recognition memory initially favors recent information but transitions to earlier information as the retention interval increases. \citet{yuan1997} further supports this by finding that temporal variations helped to oscillate judgments. Largely, since memory-based judgement is closely tied to recall \citep{begg1989} and recall is dominated by the recency effect, it follows that judgments based on recalled information will exhibit this bias. However, as time progresses, judgments may fluctuate between conflicting assessments due to shifts in memory accessibility.

This has serious implications for an individual’s belief revision. First, judgments are often influenced by an initial primacy effect. Second, as time passes, temporal variations in memory accessibility can lead to fluctuating assessments. Had Ashton and Ashton’s experimental design incorporated a delay between evidence presentation and final judgment, as is common in audit settings, their findings might have differed, as supported by prior literature on memory and judgment. This highlights that forming judgments in auditing is more complicated than some experiments have accounted for, especially when the timing and order of evidence matter. By considering priming effects and memory biases, this study aims to offer a clearer understanding of how auditors revise their beliefs.

\subsection{The Role of Power}
As outlined above, changes in an individual's beliefs depend on the initial prime, which serves as an anchor, influencing how subsequent information is processed and recalled. This suggests that the way someone is primed shapes their response to the anchor and, ultimately, the decisions they make. In an audit context, this is particularly significant, as auditors are tasked with making critical judgments that rely on their ability to assess and integrate financial information objectively. However, biases and external influences, particularly management representations, can affect how auditors interpret information, potentially compromising the accuracy of their assessments.

For instance, management may attempt to signal to auditors that their internal controls are effective or that their financial reporting is sound. This introduces the potential for auditors to be influenced by management’s framing of information. I expect that exposing auditors to a prime, such as management representations, will reduce their likelihood of critically investigating potential issues with financial information. This expectation is supported by substantial evidence showing that even minimal priming can lead individuals to seek confirmatory evidence that aligns with the primed conclusion. Additionally, because management’s assertions come from a position of power, auditors may be subtly encouraged to defer to these claims rather than challenge them.

This dynamic can be best understood through a lens of authority. Professional values, which emphasize skepticism and independent judgment, may at times compete with hierarchical authority, making auditors feel an implicit obligation to accept management’s influence. \citet{lippitt1952} established that an individual’s perception of authority plays a crucial role in organizational settings. Their research, among others, finds that individuals are more likely to adopt the attitudes, decisions, and actions of those they perceive as more powerful. This is consistent with findings from classic studies on obedience and conformity \citep{asch2016, milgram1974, burger2009}, which demonstrate that people tend to defer to authority figures, even when doing so conflicts with their own judgment or ethical considerations.

At the core of this premise is the idea of power dynamics and the mutual dependence between management and auditors \citep{casciaro2005}. External audits require auditors to interact extensively with management, creating opportunities for management to influence auditor judgment. This is primarily because auditors depend on management to provide financial data, explain accounting policies, and offer insights into business operations. Consequently, management’s control over the dissemination of information allows them to shape the auditor’s access to important facts, potentially steering the audit process in their favor, particularly if auditors do not counterbalance this influence with sufficient skepticism and independent verification. In this vein, the auditor may attempt to mitigate the constraint imposed by management by collecting more evidence or conducting additional substantive tests. However, they remain at an inherent disadvantage due to the position in which management has placed them. 

Power not only enables management to shape the information environment but also enhances the effectiveness of cognitive biases such as priming by increasing auditors’ reliance on management’s framing of disseminated information. As auditors become more dependent on management under these constrained conditions, their exposure to management’s perspectives increases, making them more susceptible to its influence and more likely to adopt them. As previously discussed, psychological research indicates that individuals are more likely to internalize information when it comes from an authoritative source \citep{lippitt1952, milgram1974}. This suggests that auditors who rely heavily on management may unconsciously align their judgements with management’s framing as an anchor for decision-making due to the pressure to defer to authority. As a result, auditors are more likely to revise their beliefs in alignment with management’s assertions.

As an example, evidence suggests that young, staff-level auditors are often “mismatched” with client management, whom they perceive as significantly more experienced and knowledgeable. Feeling inexperienced by comparison, they may be reluctant to question management’s expertise and, as a result, limit the extent to which they collect audit evidence to minimize direct interactions \citep{bennett2013}. This implies that staff-level auditors are much more likely to take management at their word, not only because they view them as a credible authority but also to avoid potentially intimidating or uncomfortable situations. This dynamic also reinforces the power imbalance, as less-experienced auditors may hesitate to challenge management’s assertions or seek additional evidence, ultimately shaping the judgments they make.

Given the potential for management’s authority to shape auditors’ perceptions through priming and the power imbalance inherent in the auditor-management relationship, the following hypotheses are proposed:

\begin{center}
    \textbf{H1a} When management priming is present, individuals will exhibit less belief revision in response to new evidence compared to when no prime is present.
\end{center}

\begin{center}
    \textbf{H1b} When management priming is present, individuals will be less conservative.
\end{center}

\begin{center}
    \textbf{H1c} When individuals are given time between the presentation of evidence and the formation of a judgment, they will revise their final belief upward.
\end{center}

\subsection{Skepticism Under Priming}
Auditors are expected to exercise professional judgment and fulfill their responsibilities with integrity, objectivity, and due care \citep{aicpa2012}. This expectation implies that auditors should critically assess all evidence and ensure that financial information is accurate, reliable, and free from material misstatement. However, individuals tend to evaluate evidence in a way that reinforces their preexisting beliefs, reacting more strongly to confirming evidence than to contradictory information \citep{gorman1986}. This suggests that an auditor’s level of professional skepticism influences how they process and integrate evidence, ultimately shaping their judgment and decision-making. Specifically, if an auditor does not approach their work with a questioning mind or critically assess audit evidence, then they would be much more susceptible to either preexisting beliefs or react more strongly to information that aligns with those beliefs. Importantly, priming can further shape this process by making certain information more accessible in memory, subtly guiding how auditors interpret subsequent evidence. When auditors are primed with management’s representations, for example, they may be more likely to subconsciously align their evaluations with the primed perspective, reinforcing confirmation bias. On the other hand, auditors with high professional skepticism may be more resistant to these effects, critically scrutinizing all evidence rather than relying on initial impressions or management framing. Therefore, understanding that mindsets play a crucial role in influencing the strength biases \citep{alexopoulos2012} is essential for examining how auditors process and respond to information during judgment and decision-making.

Research has found that auditors are prone to primacy effects, meaning they may give disproportionate weight to initial information while failing to fully integrate later information \citep{evans1993, mcmillian1993, anderson1999}. I assume that high professional skepticism will help mitigate the effect of priming by encouraging auditors to integrate late-arriving information more thoroughly. This aligns with \citet{hilton1993}, who argue that innate suspicion fosters more effortful information processing by suspending initial belief, ultimately leading to better judgment. Research has also shown that auditors who adopt an error-focused perspective tend to be more attentive to audit evidence, whether it confirms or challenges their initial assumptions \citep{mcmillian1993}. So, high professional skepticism should further reinforce this tendency, increasing auditors’ focus on detecting errors, and making them more cautious about accepting evidence at face value. 

In contrast, auditors with low professional skepticism are expected to behave systematically differently. Indeed, evidence shows that auditors with higher levels of professional skepticism tend to approach audit tasks differently than their less skeptical counterparts, even when the engagement does not inherently prompt heightened skepticism \citep{hurtt2008}. Consequently, I assume that low professional skepticism will operate in the opposite manner of high professional skepticism, that is, the effect of priming should be amplified rather than mitigated.

Given the potential for the level of professional skepticism to shape auditors’ judgments and information processing, the following hypothesis is proposed:

\begin{center}
    \textbf{H2}: High (low) professional skepticism will negate (bolster) the effect of management priming.
\end{center}


\section{Methodology}
\subsection{Manipulations and Dependent Variable}
The first manipulated variable is whether the participant receives a prime or not. Participants in the experimental condition will be exposed to a priming stimulus (i.e., a statement indicating that management is confident their internal controls are working effectively) intended to influence how they subsequently integrate additional audit evidence. In contrast, participants in the control condition will not receive any prime, allowing for a comparison of how the presence or absence of the prime affects their information processing and judgment.

The second manipulated variable is the sequence of audit evidence presented to participants, either beginning with two positive (supportive) pieces of evidence or two negative (contradictory) pieces of evidence. This manipulation is designed to test for order effects by examining whether the initial tone of the evidence—supportive or contradictory—influences participants’ subsequent judgments. By holding the content of the evidence constant across conditions and varying only the order in which it is presented, the study isolates the potential impact of early information on belief revision and risk assessment.

To test the hypotheses, we will measure participants’ initial (pre-manipulation) and final (post-manipulation) judgments to assess whether they change significantly as a function of the experimental manipulation. Specifically, we will compare the difference between initial and final judgments across conditions to determine whether management priming and professional skepticism influence participants’ assessments. We hypothesize that the change in judgment will be significantly smaller for participants who are primed, reflecting a primacy effect in which early information disproportionately shapes decision-making. This may be due to auditors’ tendency to defer to management's perceived authority. Additionally, we expect that participants with higher levels of professional skepticism will show smaller changes in judgment than those with lower levels, as skeptical individuals are less susceptible to external influence and more likely to maintain consistent evaluations.

\subsection{Measuring Professional Skepticism}
In this study, professional skepticism will be seen as a moderator and treated as a personality trait. More specifically, at its core characteristics—such as questioning mind, suspension of judgment, and a desire to search for knowledge—are assumed to vary across individual auditors. Based on this perspective, I will use the 30-item Hurtt Scale, which is specifically designed to measure the degree of professional skepticism an individual possesses. When describing the efficacy of the scale, Hurtt (2007) emphasized that the internal consistency coefficient for the administration of the 30-item scale using Cronbach’s alpha was 0.86. This suggests that the items in the scale are closely related, the scale has strong reliability, and that its items consistently capture the construct of professional skepticism.

\subsection{Participants}
81 undergraduate and graduate accounting students were recruited to participate in this experiment through their accounting courses. Serving as proxy auditors, these students engaged in a simulated audit environment designed to explore belief revision in response to management priming. To ensure a minimum level of audit knowledge, participants were required to have completed or be currently enrolled in an auditing course. 

Participants’ audit and internal control knowledge ranged from “Somewhat Knowledgeable” to “Knowledgeable.” While many participants reported discomfort with assessing internal controls, approximately half had prior experience doing so and were familiar with the relevant procedures. A majority also expressed confidence in their ability to evaluate internal control effectiveness. In addition, 100 percent of participants were accounting majors. So, despite that they may lack extensive practical auditing experience, their academic background equips them with a solid grasp of audit procedures, internal controls, and risk assessment. This makes them appropriate participants for examining judgment and decision-making within an audit-related context. Additional demographic information about participants is summarized in Table 1.

\begin{center}
    \textbf{[Insert Table 1]}
\end{center}

\subsection{Experimental Design and Procedures}
This study employs a 2x2 between-subjects experimental design. As part of this design, the experimental task in this study will replicate the judgment task developed by \citet{ashton1988}, originally used to examine how auditors respond to order effects and sequential information processing. The context involves an audit of a hypothetical client’s internal controls, with each participant asked to assume the role of an auditor responsible for assessing the effectiveness of those controls. Each participant will be given the same initial scenario and asked to provide a baseline judgment regarding the effectiveness of the client’s internal controls. Subsequently, they will receive four additional pieces of evidence, one at a time, and after each, they will be asked to update their judgment based on the information available up to that point. 

Participants will be randomly assigned to one of two experimental conditions designed to manipulate key variables of interest (i.e., priming and evidence order). For participants assigned to the priming condition, an additional detail is introduced immediately following the baseline judgment: they are informed that management feels confident about the effectiveness of their internal controls. This manipulation is designed to examine whether management's expressed confidence influences auditors’ subsequent judgments during the evidence evaluation process. If participants are not randomly assigned to the priming condition, they will receive no additional details. The design also allows for the observation of judgment changes over time based on the potential influence of order effects. Participants will be assigned to one of two evidence order conditions: either they will receive two positive pieces of evidence followed by two negative ones, or two negative pieces followed by two positive ones.

The scenario was intentionally designed with evidence that is approximately equally important apart from the explicit order effects, to ensure that the overall effectiveness of the client’s internal controls is difficult to determine. Despite this ambiguity, participants will be asked at the conclusion to assess the internal controls and indicate whether they believe they are effective or not. This final judgment is intended to determine whether the priming manipulation influenced participants’ beliefs and whether the order effects had any nominal influence on the participants final decision. Following the main task, participants will complete a post-experiment survey, which includes demographic questions and a validated measure of professional skepticism. This measure is used to assess whether an individual’s level of professional skepticism moderates the effect of the prime on belief revision or other cognitive biases.

\section{Results}
\subsection{Manipulation Check}
To verify participants’ attention to the order effects manipulation, the post-experiment questionnaire asked them to classify each piece of sequentially presented evidence as either a strength or a weakness. Response accuracy was high across all manipulation check items. Specifically, 97.1 percent of participants correctly identified the first and second items, 88.2 percent correctly identified the third item, and 95.6 percent correctly identified the fourth item. These results suggest that participants paid close attention to the nature and order of the audit evidence presented. To assess the effectiveness of the priming manipulation, participants were also asked, whether they received the prime or not, how they perceived management’s opinion about internal controls, and whether they believed the company’s internal controls would support its continued operations. For those that received the prime, participants’ responses suggest the manipulation was successful. On average, participants rated management’s opinion about internal controls as relatively strong (M = 69.5, SD = 19.4), viewed the company’s internal controls as moderately effective (M = 46.2, SD = 17.4), and expressed moderate confidence in the company’s ability to continue operations based on its internal controls (M = 54.1, SD = 21.1). Similarly, those who received the prime viewed the company’s internal controls as more effective (M = 69.5, SD = 19.4) than those who did not (M = 54.0, SD = 21.7), and they expressed greater confidence in management’s focus on improving the company’s internal controls (M = 54.1, SD = 21.1 vs. M = 39.1, SD = 24.5). These differences support the effectiveness of the priming manipulation in shaping participants’ perceptions.

\subsection{Testing H1a, H1b, and H1c}
H1a predicts that participants who receive a prime will exhibit less belief revision than those who do not. To test this hypothesis, I compare the average change in participants’ judgments—measured as the difference between baseline responses and judgments following the fourth piece of sequential evidence—across primed and unprimed conditions. As shown in Table 2, Panel A, the results reveal a nuanced pattern. In the condition where negative evidence preceded positive evidence, participants exposed to management priming demonstrated significantly less upward belief revision (0.0118) compared to unprimed participants (0.0295), consistent with H1a. However, when the sequence was reversed—positive evidence followed by negative—primed participants showed greater downward revision (-0.0545) than their unprimed counterparts (-0.0400), contradicting the hypothesis. These findings suggest that the effect of priming is not uniform. Rather than simply anchoring participants to their initial interpretations, the prime appears to interact with other task features, such as evidence order, to shape belief updating. Table 2, Panel B supports this interpretation, showing marginally significant evidence that the sequence and direction of evidence influence belief revision (F(1, 77) = 3.67, p = 0.0592). Although the main effect of the prime is not statistically significant (p = 0.5345), this supports the view that priming alone may not drive belief revision, but instead operates through interactions with contextual features of the judgment task.

The results also indicate that the prime biased participants toward more conservative belief updating—heightening their sensitivity to negative information while reducing responsiveness to positive cues. Although the prime was designed to reduce caution and promote greater acceptance of internal controls as effective, participants in the primed condition exhibited greater caution and skepticism than their unprimed counterparts. Specifically, primed participants showed greater downward revision (-0.0545 vs. -0.0400) or less upward revision (0.0118 vs. 0.0295), depending on the sequence of evidence presented. One possible explanation for this pattern could be related to a demand effect. The prime may have led participants to infer that the task required increased scrutiny or caution, prompting them to adjust their responses accordingly. Instead of reducing skepticism, the prime may have unintentionally conveyed that management’s assertion regarding control effectiveness should be critically evaluated—resulting in more conservative belief updating. Prior research has also found that a contrast effect of priming, rather than an assimilation effect, is more likely to occur when participants are aware of or recall the priming events at the time they make their judgment about the stimulus \citep{higgins1996, martin1986, lombardi1987, loersch2016}. Either way, this directly opposes H1b, which predicts that when a prime is present, participants will be less conservative. 

H1c predicts that when participants are given time between the presentation of evidence and the formation of a judgment, they will revise their final belief upward. As shown in Figure 1, participants across all conditions exhibited belief adjustment following the fourth piece of evidence. Notably, Groups A, B, and D show a rebound or upward revision in belief after a decline at Evidence 4, suggesting that participants either continued integrating information or were anchoring to earlier cues. This pattern indicates that belief updating is ongoing—even late in the sequence—and highlights the dynamic nature of judgment formation in this task. To assess whether this upward revision was statistically significant, I conducted a linear mixed-effects model (Table 2, Panel C) to account for individual-level changes in belief over time. The results indicate that belief significantly increased during the early stages of evidence presentation (Evidence 1 and Evidence 2), but not after the fourth piece of evidence or at the final judgment stage. Although there was a slight increase in belief at the final stage (Estimate = 0.0023, p = 0.5615), this change was not statistically significant, suggesting that participants placed greater weight on earlier information, or that late-stage cues lacked sufficient influence to meaningfully shift final judgments. It is also notable that in several conditions, final judgments returned to or fell below baseline levels—consistent with a recency bias or increased sensitivity to negative information presented at the end of the sequence. These mixed patterns suggest that while delayed judgment can support continued belief updating, the direction and strength of that revision may depend on the timing, order, and framing of the evidence presented.

\begin{center}
    \textbf{[Insert Table 2]}
\end{center}

\begin{center}
    \textbf{[Insert Figure 1]}
\end{center}

\subsection{Testing H2}
H2 posits that professional skepticism will moderate the effect of management priming on belief revision, whereby participants with high skepticism will exhibit less belief revision than those with low skepticism. To test this hypothesis, I conducted a two-way ANOVA examining the main and interactive effects of priming and professional skepticism on belief revision scores. In addition, I examined the mean changes in belief revision across the experimental conditions, comparing group-level differences within each level of skepticism. This analysis was followed by simple effects tests and pairwise comparisons within each level of skepticism to evaluate whether the influence of priming varied between high- and low-skepticism participants. The results, summarized in Table 3, do not support H2. Contrary to expectations, the range of mean belief revision was greater among high-skepticism participants (0.158) than among low-skepticism participants (0.097), suggesting greater—not reduced—variability in response to priming. Moreover, within the primed conditions, high-skepticism participants revised their beliefs by −0.0955 or −0.0111, while low-skepticism participants revised by only −0.0136 or 0.0375. Compared to the non-primed participants, who exhibited relatively modest belief changes (ranging from 0.0167 to −0.0141 for high-skepticism and from 0.0385 to 0.0833 for low-skepticism), the primed participants—particularly those with high skepticism—demonstrated more pronounced shifts in belief. Rather than buffering against the influence of priming, professional skepticism appears to have heightened participants' reactivity to the evidence, suggesting that highly skeptical individuals may have engaged more deeply with the task. 

Overall, Panel B shows a significant main effect of professional skepticism on belief revision (p = .0059), indicating that participants with higher skepticism revised their beliefs differently than those with lower skepticism, regardless of whether they were primed. However, the interaction between priming and skepticism was not significant (p = .374), suggesting that while skepticism influences belief revision, it does not moderate the effect of priming as hypothesized. Panel C and Panel D further support this conclusion, showing that there was no significant difference in belief revision between primed and non-primed participants within either the high-skepticism group (p = .82) or the low-skepticism group (p = .252). These findings contradict the hypothesis that professional skepticism would diminish the influence of management priming.

Figure 2 illustrates this pattern more clearly, depicting belief revision over time by group and level of professional skepticism. The figure shows that high-skepticism participants (top row) exhibited more pronounced shifts in belief—especially in Groups A and D—compared to low-skepticism participants (bottom row), whose responses were generally flatter and less variable. Notably, belief trajectories among high-skepticism participants in the primed conditions (Groups A and B) reveal sharper changes across evidence stages, reinforcing the pattern that highly skeptical individuals were more responsive to the information presented, contrary to the expectation that they would be less influenced by priming.

\begin{center}
    \textbf{[Insert Table 3]}
\end{center}

\begin{center}
    \textbf{[Insert Figure 2]}
\end{center}

\section{General Discussion and Conclusion}
Auditors are fundamentally responsible for obtaining reasonable assurance that a company’s financial statements are free from material misstatement \citep{pcaob2024}. This requires exercising professional judgment and appropriate skepticism, as well as critically evaluating the audit evidence to form a well-supported judgement on the financial statements. While prior research \citep{krishnamoorthy1999, ashton1988} shows that auditors' belief revision is especially responsive to disconfirming and negative evidence—particularly when information is presented sequentially and with a recency effect—but audit judgment must also be influenced by other contextual and individual factors, since belief revision occurs within dynamic audit environments that involve varying task demands, time pressures, and cognitive constraints.

In this study, modeled after \citet{ashton1988}, I develop theory and present supporting evidence that auditor judgments are influenced not only by the nature of the evidence itself but also by situational cues (such as managerial framing) and individual factors that are embedded within the audit process. The \citet{ashton1988} study is noteworthy because it demonstrates that auditors’ belief revision is systematically influenced by the order in which information is presented, highlighting the presence of recency effects in professional judgment. However, the study primarily focused on internal validity, limiting the extent to which its findings can be generalized to more complex or naturalistic audit settings. The results of my study indicate that the direction of evidence, individual characteristics, and external stimuli interact to shape the extent to which individuals revise their beliefs. 

Prior research has established the importance of considering both cognitive biases and social influences in understanding how auditors form and revise judgments \citep{loersch2016, bargh1996, zajonc1980, milgram1974}. Additionally, prior literature suggests that priming, especially in the context of power imbalances, may suppress professional skepticism and limit belief revision \citep{hammersley2010, bennett2013, casciaro2005}. However, I found the opposite: rather than suppressing belief revision, management priming appeared to heighten auditors’ sensitivity to new information. This unexpected result suggests that priming may not always lead to reduced skepticism or conformity with management’s position. Instead, it may activate a heightened awareness or critical stance, particularly when auditors perceive the prime as a signal requiring closer scrutiny. This finding challenges assumptions in prior literature and highlights the need to further examine the nuanced ways in which contextual cues influence auditor judgment. Prior research has also determined that auditors’ level of professional skepticism plays a critical role in how they process information under priming, with highly skeptical auditors more likely to critically evaluate evidence and resist biased influence (such as management framing) while those with low skepticism are more susceptible to confirmation bias and more strongly affected by priming effects \citep{hurtt2008, alexopoulos2012, mcmillian1993}. However, I found the opposite here as well: rather than auditors with low professional skepticism reacting more strongly to the prime, it was those with high skepticism who appeared more influenced by it. This unexpected result suggests that highly skeptical auditors may have interpreted the prime as a cue requiring closer scrutiny, potentially heightening their sensitivity to subsequent evidence rather than dampening it. 

Although participants responded in the opposite direction of what was initially predicted, the findings still hold important implications for understanding how auditors interpret and revise their judgments in realistic and complex audit environments. First, traditional auditing theory treats professional skepticism as a protective mechanism against bias \citep{hurtt2008}, but my findings challenge this view that highly skeptical auditors can be more influenced by external cues like priming, which highlights a more complex relationship between skepticism and judgement than originally thought. Second, my findings suggest that professional skepticism alone may not be sufficient to guard against cognitive biases or the subtle influence of external stimuli. As a result, auditors should be mindful of these limitations and consider implementing additional safeguards—such as structured reflection, independent review, or incubation—to mitigate bias beyond simply relying on skepticism. Firms should also be aware of how management representations and framing during the audit planning and audit process could unintentionally influence auditors, even those that are highly skeptical. Third, my findings support the view that belief revision is a dynamic process and not purely evidence driven, so future experimental work should account for temporal spacing, authority cues, and auditor mindset as interacting factors to create a more valid test of auditor judgement. 

This study contributes to the auditing literature by offering new insights into how contextual cues—such as priming and management influence—interact with individual auditor characteristics like professional skepticism to shape belief revision and judgment. It challenges the assumption that skepticism alone is sufficient to guard against bias and highlights the need for greater attention to cognitive and social dynamics in audit environments. Additionally, the findings expand our understanding of the mechanisms through which authority and framing can affect professional decision-making, providing practical implications for auditor training, supervision, and the design of audit procedures.

\vfill
\pagebreak{}
\begin{spacing}{1.0}
\bibliographystyle{apalike}
\bibliography{PS11_Bermudez}
\addcontentsline{toc}{section}{References}
\end{spacing}

\vfill
\pagebreak{}
\clearpage

%========================================
% FIGURES AND TABLES 
%========================================
\section*{Figures and Tables}\label{sec:figTables}
\addcontentsline{toc}{section}{Figures and Tables}
%----------------------------------------
% Table 1
%----------------------------------------

\begin{table}[ht]
\centering
\caption{Participant Demographics and Background Knowledge}
\begin{tabularx}{\textwidth}{l l r r r r r r r}
\toprule
Variable & Response & Unique & Missing & Mean & SD & Min & Median & Max \\
\midrule
Duration & & 73 & 0\% & 11.1 & 6.9 & 2.0 & 9.4 & 54.4 \\
\addlinespace
Gender & Male & 45 & 56.2\% & & & & & \\
& Female & 33 & 41.2\% & & & & & \\
& Prefer Not to Say & 2 & 2.5\% & & & & & \\
\addlinespace
Graduate & Undergraduate & 76 & 95.0\% & & & & & \\
& Graduate & 4 & 5.0\% & & & & & \\
\addlinespace
Skepticism & High & 39 & 48.8\% & & & & & \\
& Low & 41 & 51.2\% & & & & & \\
\addlinespace
Auditing Knowledge & Very Little & 4 & 5.0\% & & & & & \\
& Somewhat Knowledgeable & 34 & 42.5\% & & & & & \\
& Knowledgeable & 38 & 47.5\% & & & & & \\
& Very Knowledgeable & 4 & 5.0\% & & & & & \\
\addlinespace
Auditing Internship & Yes & 19 & 23.8\% & & & & & \\
& No & 61 & 76.2\% & & & & & \\
\addlinespace
IC Knowledge & Very Little & 5 & 6.2\% & & & & & \\
& Somewhat Knowledgeable & 40 & 50.0\% & & & & & \\
& Knowledgeable & 30 & 37.5\% & & & & & \\
& Very Knowledgeable & 5 & 6.2\% & & & & & \\
\bottomrule
\end{tabularx}
\end{table}

%----------------------------------------
% Table 2
%----------------------------------------

\begin{table}[ht]
\centering
\caption{Tests of H1a, H1b, H1c}
\label{tab:hypothesis_tests}
\begin{tabular}{lrrrrr}
\toprule
\multicolumn{5}{l}{\textbf{Panel A: Mean Belief Revision by Condition}} \\
\midrule
Condition & $n$ & Mean & SD & Difference \\
Prime ($++$, $--$) & 22 & $-0.0545$ & 0.146 & \\
No Prime ($++$, $--$) & 20 & $-0.0400$ & 0.190 & \textbf{-0.0145} \\
Prime ($--$, $++$) & 17 & $0.0118$ & 0.154 & \\
No Prime ($--$, $++$) & 22 & $0.0295$ & 0.144 & \textbf{-0.0177} \\
\addlinespace
\multicolumn{5}{l}{\textbf{Panel B: ANOVA on Belief Revision by Prime and Evidence Direction}} \\
\midrule
Variation Source & Sum of Squares & df & Mean Square & F & $p$ \\
Prime & 0.0098 & 1 & 0.00985 & 0.389 & 0.5345 \\
Direction & \textbf{0.0928} & 1 & 0.09278 & \textbf{3.667} & \textbf{0.0592} \\
Prime $\times$ Direction & 0.0001 & 1 & 0.00005 & 0.002 & 0.9638 \\
Error & 1.9485 & 77 & 0.02531 & & \\
\addlinespace
\multicolumn{5}{l}{\textbf{Panel C: Fixed Effects Table}} \\
\midrule
Term & Estimate & $t$ value & $p$ & Sig. \\
Prime & 0.130 & 3.312 & 0.00101 & \textbf{**} \\
Evidence 1 & 0.239 & 6.101 & 2.39e-09 & \textbf{***} \\
Evidence 2 & 0.273 & 6.973 & 1.21e-11 & \textbf{***} \\
Evidence 3 & 0.036 & 0.930 & 0.35307 & \\
Evidence 4 & $-0.055$ & $-1.395$ & 0.16389 & \\
Final & 0.023 & 0.581 & 0.56152 & \\
\bottomrule
\end{tabular}

\vspace{0.5em}
\raggedright
\footnotesize{\textit{Note.} Bolded values reflect contrasts of interest. $^{*}p < .05$, $^{**}p < .01$, $^{***}p < .001$.}
\end{table}

%----------------------------------------
% Table 3
%----------------------------------------
\begin{landscape}
\begin{table}[ht]
\centering
\caption{Tests of H2}
\label{tab:h2_tests}
\begin{tabular}{llrrrrrr}
\toprule
\multicolumn{7}{l}{\textbf{Panel A: Mean Belief Revision by Group and Skepticism Level}} \\
\midrule
Skepticism & Condition & $n$ & Mean & SD & Difference (max--min) & \\
High & Prime ($++$, $--$) & 11 & $-0.0955$ & 0.180 & & \\
High & No Prime ($++$, $--$) & 9 & $-0.0167$ & 0.135 & & \\
High & Prime ($--$, $++$) & 9 & $-0.0111$ & 0.180 & & \\
High & No Prime ($--$, $++$) & 11 & $0.0141$ & 0.156 & \textbf{0.158} & \\
Low & Prime ($++$, $--$) & 11 & $-0.0385$ & 0.156 & & \\
Low & No Prime ($++$, $--$) & 8 & $0.0295$ & 0.156 & & \\
Low & Prime ($--$, $++$) & 13 & $0.0375$ & 0.125 & & \\
Low & No Prime ($--$, $++$) & 9 & $0.0833$ & 0.154 & \textbf{0.097} & \\
\addlinespace
\multicolumn{7}{l}{\textbf{Panel B: ANOVA on Belief Revision by Prime and Skepticism}} \\
\midrule
Variation Source & Sum of Squares & df & Mean Square & $F$ & $p$ & Sig. \\
Prime & 0.0098 & 1 & 0.00985 & 0.414 & 0.52177 & ns \\
Skepticism & \textbf{0.1911} & 1 & 0.19111 & \textbf{8.036} & \textbf{0.00585} & \textbf{**} \\
Prime $\times$ Skepticism & 0.0190 & 1 & 0.01905 & 0.801 & 0.37361 & ns \\
Error & 1.8312 & 77 & 0.02378 & & & \\
\addlinespace
\multicolumn{7}{l}{\textbf{Panel C: ANOVA on Belief Revision by Group Within Skepticism Level}} \\
\midrule
Skepticism & Source & Sum of Squares & df & Mean Square & $F$ & $p$ & Sig. \\
High & Prime & 0.0016 & 1 & 0.00156 & 0.053 & 0.82 & ns \\
& Error & 1.128 & 38 & 0.02969 & & & \\
Low & Prime & 0.0244 & 1 & 0.0244 & 1.354 & 0.252 & ns \\
& Error & 0.7028 & 39 & 0.01802 & & & \\
\addlinespace
\multicolumn{7}{l}{\textbf{Panel D: Pairwise Comparisons}} \\
\midrule
Skepticism & Contrast & Estimate & $p$ & Sig. & \\
High & No Prime $-$ Prime & $-0.013$ & 0.82 & ns & \\
Low & No Prime $-$ Prime & $0.0489$ & 0.252 & ns & \\
\bottomrule
\end{tabular}

\vspace{0.5em}
\raggedright
\footnotesize\textit{Note.} Bolded values indicate statistically significant effects. $^{*}p < .05$, $^{**}p < .01$. “ns” indicates not significant.
\end{table}
\end{landscape}

%----------------------------------------
% Figure 1
%----------------------------------------

\begin{figure}[ht]
    \centering
    \includegraphics[width=0.8\textwidth]{Rplot02.png}
    \caption{Belief Revision Over Time by Experimental Group. Black lines = group mean; gray lines = individual responses.}
    \label{fig:belief_revision_groups}
\end{figure}

%----------------------------------------
% Figure 2
%----------------------------------------

\begin{figure}[ht]
    \centering
    \includegraphics[width=0.8\textwidth]{belief.png}
    \caption{Belief Revision Over Time by Group and Professional Skepticism Level. Black lines represent group means; gray lines show individual response trajectories.}
    \label{fig:belief_revision_skepticism}
\end{figure}


\end{document}
