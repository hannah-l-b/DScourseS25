\documentclass{article}
\usepackage{graphicx} % Required for inserting images


% Set page size and margins
% Replace `letterpaper' with `a4paper' for UK/EU standard size
\usepackage[letterpaper,top=2cm,bottom=2cm,left=3cm,right=3cm,marginparwidth=1.75cm]{geometry}

% Useful packages
\usepackage{amsmath}
\usepackage{tabularray}
\usepackage{float}

\title{PS9 - ECON 5253}
\author{Hannah Bermudez}
\date{April 2025}

\begin{document}

\maketitle

\section{Q7}

\begin{itemize}
    \item What is the dimension of your training data?
    \begin{itemize}
        \item The training data has 74 rows and 404 columns.
    \end{itemize}
    
    \item How many more X variables do you have than in the original housing data?
    \begin{itemize}
        \item The original housing dataset had 14 variables, 13 of which were predictors (X variables). This suggests that the training data has 61 more predictor variables than the original dataset.
    \end{itemize}
\end{itemize}

\section{Q8 - Estimating LASSO}

\begin{itemize}
    \item What is the optimal value of lambda?
    \begin{itemize}
        \item The optimal value of lambda is: 0.00222
    \end{itemize}

    \item What is the in-sample RMSE?
    \begin{itemize}
        \item The in-sample RMSE is: 0.138
    \end{itemize}

    \item What is the out-of-sample RMSE?
    \begin{itemize}
        \item The out-of-sample RMSE is: 0.184
    \end{itemize}
\end{itemize}

\section{Q9 - Estimating Ridge}

\begin{itemize}
    \item What is the optimal value of lambda?
    \begin{itemize}
        \item The optimal value of lambda is: 0.0000000001
    \end{itemize}

    \item What is the in-sample RMSE?
    \begin{itemize}
        \item The in-sample RMSE is: 0.140
    \end{itemize}

    \item What is the out-of-sample RMSE?
    \begin{itemize}
        \item The out-of-sample RMSE is: 0.181
    \end{itemize}
\end{itemize}

\section{Q10}

\begin{itemize}
    \item Would you be able to estimate a simple linear regression model on a data set that had more columns than rows?
    \begin{itemize}
        \item You cannot estimate a simple linear regression model (using OLS) when there are more columns than rows. This is due to multicollinearity in the X matrix. 
    \end{itemize}
    
    \item Using the RMSE values of each of the tuned models in the previous two questions, comment on where your model stands in terms of the bias-variance trade-off.

    \begin{itemize}
        \item For the LASSO model, the in-sample RMSE was 0.138 and the out-of-sample RMSE was 0.184, while the Ridge model had an in-sample RMSE of 0.140 and an out-of-sample RMSE of 0.181. The LASSO model fits the training data slightly better (lower in-sample RMSE) but performs slightly worse on the test data (higher out-of-sample RMSE), indicating it may have higher variance. Conversely, the Ridge model sacrifices a bit of in-sample fit for improved generalization, suggesting it introduces slightly more bias but achieves lower variance overall. Therefore, the Ridge model may be better balanced in terms of the bias-variance trade-off in this case, as its smaller out-of-sample error indicates better predictive performance on new data. 
    \end{itemize}
\end{itemize}

\end{document}
